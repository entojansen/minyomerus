%% Submissions for peer-review must enable line-numbering 
%% using the lineno option in the \documentclass command.
%%
%% Preprints and camera-ready submissions do not need 
%% line numbers, and should have this option removed.
%%
%% Please note that the line numbering option requires
%% version 1.1 or newer of the wlpeerj.cls file, and
%% the corresponding author info requires v1.2

\documentclass[fleqn,10pt,lineno]{wlpeerj} % for journal submissions
% \documentclass[fleqn,10pt]{wlpeerj} % for preprint submissions

\usepackage{gensymb}
\usepackage{wasysym}
\usepackage{textgreek}
\usepackage{textcomp}
\newcommand{\td}{\textdegree~}
\newcommand{\x}{$\times$~}

\title{Descriptions of four new species of \textit{Minyomerus} Horn, 1876 (Coleoptera: Curculionid\ae), with notes on their distribution and phylogeny}

\author[1]{M. Andrew Jansen}
\author[2]{Nico M. Franz}
\affil[1]{School of Life Sciences, 427 E Tyler Mall, PO Box 874501, Tempe, AZ 85287}
\affil[2]{ASU Natural History Collections, 734 W Alameda Dr, Tempe, AZ 85282}
\corrauthor[1]{M. Andrew Jansen}{majanse1@asu.edu}

% \keywords{Keyword1, Keyword2, Keyword3}

\begin{abstract}
Dummy abstract text. Dummy abstract text. Dummy abstract text. Dummy abstract text. Dummy abstract text. Dummy abstract text. Dummy abstract text. Dummy abstract text. Dummy abstract text. Dummy abstract text. Dummy abstract text.
\end{abstract}

\begin{document}

\flushbottom
\maketitle
\thispagestyle{empty}

\section*{Introduction}
	The weevil genus \textit{Minyomerus} Horn, 1876 is currently assigned to the tribe Tanymecini Lacordaire, 1863 (Curculionidae: Entiminae). 
	The genus was recently revised and includes a total of 17 described species distributed throughout the desert and plains regions of North America.
	Since the publication of the monographic revision of \textit{Minyomerus}, we have discovered four additional undescribed species, known to us only from limited numbers of specimens, yet falling neatly into the generic delimitation of \textit{Minyomerus} \textit{sensu} Jansen and Franz 2015.
	Here we describe these species and provide images of the holotypes and of dissected genitalia for the purpose of identification.
	We additionally conduct a morphological phylogenetic analysis of the genus to clarify the placement of these new taxa within \textit{Minyomerus}, based on the analysis provided in our previous work.
	An emended identification key to the species of \textit{Minyomerus} is given along with information on the geographic distribution of the herein described species.
	Where possible, we make note of host-plant records and briefly consider the historical biogeographic relationships of the new species.
	A more extensive discussion of the habits, distribution, and delimitation of the genus (and all of its constituent species) can found in the generic revision.
	
\section*{Materials and methods}
	The keen reader will note a certain similarity between the methods used in this manuscript with those of Jansen and Franz 2015. 
	Rather than merely referencing the monograph as a source of inspiration, we feel it is better to provide a fully transparent and updated accounting of the methods used to generate the species descriptions, distribution maps, and phylogeny herein.
	Most notably, we retain the format of the species descriptions, whereby we emphasize only those characters that vary significantly from the generic description, as explained below.
	\paragraph{Natural history observations} 
		Members of the genus \textit{Minyomerus} may be found on a variety of host plants, as indicated following each description.
		While many species appear to be generalists, the adults are consistently observed on the leaves and branches of the host feeding on the leaf tissue; all other life stages remain unknown.
		Members of this genus are commonly found in deserts throughout western North America, including the Mojave, Sonoran, Chihuahuan, and Great Basin Deserts. 
		Furthermore, the range of the genus extends through the semi-arid regions of the Great Plains, the Colorado Plateau, and Baja California, Mexico. 
		The adults are flightless, as the hind wings and associated flight structures of all species are either greatly reduced or not readily apparent in dissection.

	\paragraph{Acquisition of museum specimens} 
		Georeferencing of localities was performed with Google Earth (Google Inc. 2009–2015) and reported in decimal degrees.
		Taxonomic names for associated host plants are used in accordance with Munz and Keck (1973) and SEINet (2017).
		The existing set of \textit{Minyomerus} specimens used in the revision of Jansen and Franz 2015 was supplemented with material from the following collections, using the codens of Arnett et al. (1993):
		
		\begin{description}[itemsep=-1ex]
			\item[\texttt{CMNC}] Canadian Museum of Nature Collection, Ottawa, Ontario, Canada
			\item[\texttt{TAMU}] Texas A \& M University, College Station, Texas, USA
			\item[\texttt{USNM}] National Museum of Natural History, Washington, D.C., USA
		\end{description}
		
	\paragraph{Morphological analysis}
		Our systematic and descriptive approach is complementary to Jansen and Franz 2015, which in turn follows Franz (2010a, 2010b, 2012).
		Terminology for exterior morphology is in general accordance with Torre-Bueno (Nichols 1989).
		Additional morphological terms specific to (entimine) weevils were used as follows:
			Ting (1936) and Morimoto and Kojima (2003) for mouthparts;
			Thompson (1992) for tibial apices and abdominal segments;
			and Oberprieler et al. (2014) and Howden (1995) for male and female terminalia.
		
		Measurements were taken with a Leica M205 C stereomicroscope and associated software, Leica Application Suite (LAS), version 4.1.0.
		Overall body length and width were measured in dorsal view as the maximum distance between the rostral and elytral apices, and the maximum width of both elytra, respectively.
		Rostral length was measured in dorsal view as the distance between the epistomal apex and the anterior margin of the eyes.
		Rostral width was measured in dorsal view as the maximum distance between the dorsal margins of the rostrum near the point of antennal insertion.
		Pronotal length was measured in dorsal view as the length along the midline between the anterior and posterior margins.
		The width of an individual elytron was measured in dorsal view as the maximum distance between the lateral margin and the elytral suture.
		Other length and width measurements were also performed in dorsal orientation, using the maximum length and width of the corresponding structure (profemur, protibia, elytron, and aedeagus).
		Images of mouthparts and terminalia were produced with the Leica microscope equipment, while habitus photographs were created with a Visionary Digital Passport II sytem using a Canon EOS Mark 5D II camera.

		Herein recognized species of \textit{Minyomerus}  were delimited through application of the phylogenetic species concept \textit{sensu} Wheeler and Platnick (2000).
		Species descriptions are in alphabetical order, rather than phylogetic order, for ease of use.
		The species descriptions in this manuscript represent unique and complementary accounts of the character states observed in each species, including their variability, but except those characters invariant within the genus as specified in the higher-level description of \textit{Minyomerus}.
		Likewise, descriptions of males emphasize characters that are variable and sufficiently different from those of the females to merit recognition.
		The key to identifying species of \textit{Minyomerus} is arranged with emphasis being placed on the most readily observable diagnostic characters.
		This manuscript is arranged with the species descriptions appearing first, followed by the key to species, and then by the phylogenetic results.
		
	\paragraph{Phylogenetic analysis}
		The morphological cladistic analysis includes 26 terminal taxa; with 21 ingroup and 5 outgroup terminals.
		Following resolution of species identities and boundaries, the ingroup terminals were represented by six species previously assigned to Minyomerus sec. O’Brien and Wibmer (1982), Piscatopus griseus sec. Sleeper (1960), and ten newly recognized species.
		The lack of prior phylogenetic analyses of Minyomerus and closely related lineages made it necessary to sample outgroups fairly broadly while remaining focused on North American lineages that are are putative close relatives of the ingroup (Nixon and Carpenter 1993).
		The tribe Tanymecini [non-focal] is cosmopolitan, with the subtribe Tanymecina [non-focal] containing the majority of New World species diversity in the tribe (Alonzo-Zarazaga and Lyal 1999).
		Accordingly, four outgroup terminals are represented by species belonging to separate genera in the Tanymecina [non-focal]; viz. Pandeleteius cinereus (Horn, 1876) [non-focal], Pandeleteinus subcancer Howden, 1969 [non-focal], Isodrusus debilis Sharp, 1911 [non-focal], and Isodacrys buchanani Howden, 1961 [non-focal].
		Because generic relationships in the Tanymecini [non-focal] remain unresolved, it was deemed prudent to find a relatively far-removed taxon to root the cladogram that would nevertheless display states that are applicable to the ingroup for all characters under consideration (Rieppel 2007; Franz 2014).
		For this purpose we select the North American species Sitona californicus (Fahraeus, 1840) [non-focal], representing the tribe Sitonini Gistel, 1856 [non-focal].
		The character matrix was edited and phylogenetic results reviewed and reconnected to the input using the WinDada and WinClados interfaces of WinClada (Nixon 2002).
		The character sequence follows that of the taxonomic descriptions.
		The most parsimonious tree and character state optimizations were inferred under parsimony using NONA (Goloboff 1999).
		An unconstrained heuristic search of tree space for the 22-terminal matrix was conducted using the commands: hold 100001, mult*1000, hold/100, with mult*max* selected.
		Bootstrap support was inferred in WinClada using the parameters of 1000 replications, hold 1000, hold/100, mult*10, “Don’t do max*”, and “Save consensus”.
		Finally, Bremer support values (Bremer 1994) were calculated in NONA using the commands hold 20000, suboptimal 20, and bsupport 20.
	\paragraph{Species distribution modeling}
		We used the modeling program Maxent, Version 3.4, to generate habitat models for the species of Minyomerus [JF 2015] (Figs 50–52) based on the documented occurrence records (Phillips et al. 2004, 2006; Elith et al. 2011).
		The default settings “Max number background points”, set to 100,000, and “Iterations” set to 10 and with cross-validation, were applied to leverage all available locality data.
		However, no models could be created for one species a single documented locality.
		We selected 19 bioclimatic variables and elevation as Environmental Layers in Maxent, obtained from WorldClim (see Hijmans et al. 2005 and the associated website for further detail on variables).
		The layers were downloaded by tile (zones 11–13 and 21–23), with a 30 arc-second resolution (projected using WSG 84) to provide adequate coverage of the full distribution of Minyomerus [JF2015].
		These tiles were assembled, by layer, into composite maps of six tiles each, using QGIS, Version 2.8.1 ‘Wien’, prior to species distribution modeling (Quantum GIS Development Team 2015).
		After modeling, the raster files of the predictive probabilities were imported into GIS.
		Each file was designated a specific color, and each pixel in the raster grid was assigned a linearly interpolated saturation of that color, with increasing saturation denoting an increased probability of successful prediction of species presence at that point.
		Pixels with a value below 0.50 were rendered transparent so that the model only shows regions with a greater than 50\% chance of successful prediction.
		The raster files were clipped to remove extraneous predicted regions based on: (1) predictive probability (i.e., removing large areas with only transparent pixels) and (2) geographic extent (accounting for endemicity).
		For instance, species endemic to Baja California Sur, Mexico, do not require a predictive model for bioclimatically similar habitats in Canada.
		The actual (documented) occurrence records are laid over the modeled habitat ranges as colored circles on the respetive maps (Figs 50–52), along with vector layers of country and state borders (Hijmans et al. 2012).
\section*{Descriptions of new species}
	\subsection*{\textit{Minyomerus ampullaceus} - sp. n.}
		\subsubsection*{Diagnosis}
			\#
		\subsubsection*{Description of female}
			\paragraph{Habitus}
				Length \# mm, width \# mm, length/width ratio \#, widest at anterior \# of elytra.
				Integument orange-brown to black. 
				Scales with variously interspersed colors ranging from slightly off-white to beige to yellow. 
				Setae recumbent to sub-recumbent, white to brown in color.
			\paragraph{Manidbles}
				Partially covered with white, slightly opalescent scales, with 3 longer setae, and 1 shorter seta between these.
			\paragraph{Rostrum}
				Length \# mm, anterior portion \# broader than long, rostrum/pronotum length ratio \#, rostrum length/width ratio \#.
				Separation of rostrum from head generally obscure. 
				Dorsal outline of rostrum nearly square, anterior half of dorsal surface mesally concave, posterior half coarsely but shallowly punctate to rugose. 
				Rostrum in lateral view nearly square; apical margin broadly bisinuate and emarginate, with 2 pairs of large vibrissae. 
				Nasal plate defined by Y-shaped, impressed lines, convex, integument partially covered with white scales.
				Margins of mandibular incision directed \#\td outward dorsally in frontal view. 
				Ventrolateral sulci strongly defined, beginning as a narrow sulcus posteriad of insertion point of mandibles, running parallel to scrobe, terminating in a ventral fovea.
			\paragraph{Antennae}
				Small tooth formed by overhanging dorsal margin of scrobe directly ventrad of margin of eye.
				Scape extending to posterior 1/3 of eye.
				Funicular segments V-VII and club missing.
			\paragraph{Head}
				Eyes globular, anterodorsal margin of each eye feebly impressed, posterior margin elevated from lateral surface of head; eyes separated in dorsal view by 4\x their anterior-posterior length, set off from anterior prothoracic margin by 1/3 of their anterior-posterior length. 
				Head without any transverse post-ocular impression.
			\paragraph{Pronotum}
				Length/width ratio \#; widest near midpoint. 
				Anterior margin slightly arcuate, lateral margins curved and widening into a bulge just anteriad of midpoint of pronotum, posterior margin straight, with a slight mesal incurvature. 
				Pronotum in lateral view with setae that reach beyond anterior margin by 1/2 of their length; these setae becoming evenly longer and more erect laterally, reaching a maximum length equal to 1/2 of length of eye. 
				Anterolateral margin with a reduced tuft of 6-7 post-ocular vibrissae present, emerging near ventral 1/2 of eye, and stopping just below ventral margin of eye; vibrissae sub-equal in length at 1/3 of anterior-posterior length of eye, except for three vibrissae achieving a maximum length similar to anterior-posterior length of eye.
			\paragraph{Scutellum}
				Exposed, margins straight.
			\paragraph{Pleurites}
				Metepisternum hidden by elytron.
			\paragraph{Thoracic sterna}
				Mesocoxal cavities separated by 1/4\x width of mesocoxal cavity. 
				Metasternum with transverse sulcus not apparent; metacoxal cavities widely separated by ca. 2\x their width.
			\paragraph{Legs}
				Profemur/pronotum length ratio \#; profemur with distal 1/5 produced ventrally as a rounded projection covering tibial joint; condyle of tibial articulation occupying 4/5 of distal surface and 1/5 length of femur. 
				Protibia/profemur length ratio \#; protibial apex with ventral setal comb recessed in an incurved groove; mucro present as a large, black, sub-triangular, medially-projected tooth, which is approximately equilateral and whose sides are subequal in length to surrounding setae. 
				Protarsus with tarsomere III 1.25\x as long as II; wider than long. 
				Metatibial apex with almond shaped convexity ringed by 10 short, spiniform setae.
			\paragraph{Elytra}
				Length/width ratio \#; widest at anterior 1/3; anterior margins jointly 1.75\x wider than posterior margin of pronotum and strongly produced dorsally from margin of pronotum; lateral margins evenly rounded until posterior 1/3, more strongly rounded and converging thereafter. 
				Posterior declivity angled at nearly 85\td to main body axis. Elytra with 10 complete striae; striae shallow; punctures faint beneath appressed scales, separated by 5-7\x their diameter; intervals very slightly elevated.
			\paragraph{Abdominal sterna}
				Ventrite III anteromesally incurved around a fovea located mesally on anterior margin, posterior margin elevated and set off from IV along lateral 1/3s of its length. 
				Sternum VII mesally 1/2\x as long as wide; anterior margin weakly curved.
			\paragraph{Tergum}
				Pygidium (tergum VIII) sub-conical; posterior margin emarginate; medial 1/3 of anterior 3/5 of pygidum less sclerotized.		
			\paragraph{Sternum VIII}
				Anterior laminar edges each incurved forming a 115\td angle with lateral margin, this angle distinctly sclerotized; posterior 1/2 of lamina porose throughout, laminar arms more sclerotized medially; posterior edge evenly, moderately arcuate.
			\paragraph{Ovipositor}
				Coxites in dorsal view slightly longer than broad, with a medial region that is weakly sclerotized.
			\paragraph{Spermatheca}
				Comma-shaped; collum expanded to form a long, cylindrical projection, subequal in length to ramus, 1/3\x width of corpus, angled at 45\td to corpus, apically with a reduced hood-shaped projection; ramus elongate, bulbous, slightly wider than thickness of corpus, basally constricted to form a short stalk; corpus not greatly swollen; cornu sub-equal in length to corpus and collum, recurved distally to form in inner angle of 60\td to corpus, straight and gradually narrowing along basal 2/3, with apical 1/3 abruptly narrowed, angled at 45\td to coprus, and tapering to a slight knob.
		\subsubsection*{Description of male}
			Male not available or known.
		\subsubsection*{Comments}
			Due to the limited number of specimens of this species, dissections of mouthparts could not be performed.
		\subsubsection*{Etymology}
			Named in reference to the shape of the body in dorsal view, which appears bottle-shaped due to the large elytra and comparatively cylindrical pronotum; ampullaceus = flasklike; Latin adjective \citep{brown1956}.
		\subsubsection*{Material examined}
			\paragraph{Holotype}
				\female~``Carlsbad, N.M.; Geococcyx calif; 144640'' (\texttt{\textbf{USNM}}).
		\subsubsection*{Distribution}
			\#
		\subsubsection*{Natural history}
			No host plant associations have been documented.
			The label indicates "Geococcyx calif"; this is presumably a reference to \textit{Geococcyx californianus} (Lesson, 1829) [Cuculidae], the Greater Roadrunner, although it is unclear if the specimen was found on or near one of these birds (either living, dead, or in a nest).
			\textit{Minyomerus} species are only known to be phytophagous, not parasitic, phoretic, or necrophagous, so we believe that this specimen was most likely found in a nest, and was present there only incidentally because the nest was constructed in the host plant of this specimen \citep{jansen2015}.
			It is unknown whether this species is parthenogenetic.
			
	\subsection*{\textit{Minyomerus franko} - sp. n.}
		\subsubsection*{Diagnosis}
			\#
		\subsubsection*{Description of female}
			\paragraph{Habitus}
				Length \# mm, width \# mm, length/width ratio \#, widest at anterior 1/3-1/4 of elytra.
				Integument orange-brown to black. 
				Scales with variously interspersed colors ranging from slightly off-white or beige to manila/tan to dark coffee brown, in some specimens appearing semi-translucent (in others opaque). 
				Setae linear to slightly apically explanate, appearing minutely spatulate, sub-recumbent to sub-erect, white or brown in color.
			\paragraph{Manidbles}
				Covered with white scales, with 3 longer setae, and 1-2 shorter setae between these.
			\paragraph{Maxillae}
				Cardo bifurcate at base with an inner angle typically between 90–120\td, arms of equal length, inner (mesal) arm nearly 1.5\x thicker than outer arm, both arms of bifurcation equal in length to apically outcurved arm, glabrous. 
				Stipes sub-quadrate, roughly equal in length to each bifurcation of cardo, with a single lateral seta. 
				Galeo-lacinial complex nearly extending to apex of maxillary palpomere II; complex mesally membranous, laterally sclerotized, with sharp demarcation of sclerotized region separating palpiger from galeo-lacinial complex; setose in membranous area just adjacent to sclerotized
region, setae covering 2/3 of dorsal surface area; dorsally with 7 apicomesal lacinial teeth; ventrally with 4 reduced lacinial teeth. 
				Palpiger with a single lateral seta, otherwise glabrous and evenly sclerotized throughout.
			\paragraph{Maxillary palps}
				I apically oblique, apical end forming a 45\td angle with base, with 2 apical setae; II sub-cylindrical, with 1 apical seta.
			\paragraph{Labium}
				Prementum roughly trapezoidal; apical margins angulate, ventral margin gently sinuate, dorsal margin straight; lateral margins feebly incurved near posterior margin; basal margin arcuate.
				Labial palps 3-segmented, I with apical 2/3 projecting beyond margin of prementum, exceeding apex of ligula; III slightly longer than II.
			\paragraph{Rostrum}
				Length \# mm, anterior portion \# broader than long, rostrum/pronotum length ratio \#, rostrum length/width ratio \#.
				Separation of rostrum from head generally obscure. 
				Dorsal outline of rostrum sub-rectangular, anterior half of dorsal surface feebly impressed, posterior half coarsely but shallowly punctate to rugose. 
				Rostrum in lateral view nearly square; apical margin bisinuate and emarginate , with 2 large vibrissae. 
				Nasal plate defined by broad, V-shaped, shallowly impressed lines, anteromesally slightly convex, integument partially covered with white scales. 
				Margins of mandibular incision directed \#\td outward dorsally in frontal view. 
				Ventrolateral sulci weakly defined (or entirely absent in some specimens) as a broad concavity dorsad of insertion point of mandibles, running parallel to scrobe, becoming flatter posteriorly and disappearing ventrally.
				Dorsal surface of rostrum with  short, linear, median fovea.
				Rostrum ventrally lacking sulci at corners of oral cavity.
			\paragraph{Antennae}
				Small tooth formed by overhanging dorsal margin of scrobe anterior to margin of eye by 1/5 of length of eye.
				Scape nearly extending to posterior 1/4 of eye.
				Terminal funicular antennomere lacking appressed scales, having instead a covering of apically-directed pubescence with interspersed sub-erect setae.
				Club nearly 3\x as long as wide.
			\paragraph{Head}
				Eyes globular to slightly elongate, slanted ca. 35\td antero-ventrally; eyes separated in dorsal view by 4\x their anterior-posterior length, set off from anterior prothoracic margin by 1/3 of their anterior-posterior length. 
				Head without any transverse post-ocular impression.
			\paragraph{Pronotum}
				Length/width ratio \#; widest near anterior 1/3, between anterior constriction and midpoint. 
				Anterior margin arcuate, lateral margins curved and widening into a slight bulge just anteriad of midpoint of pronotum, posterior margin straight, with a slight mesal incurvature. 
				Pronotum in lateral view with setae that reach just beyond anterior margin, angled laterally at 45-80\td to longitudinal axis, and strikingly long; these setae becoming evenly longer and more angled laterally, reaching a maximum length nearly equal to length of eye. 
				Anterolateral margin with a reduced tuft of 5 post-ocular vibrissae present, emerging near ventral 1/2 of eye, and stopping just below ventral margin of eye; vibrissae sub-equal in length at 1/3\x anterior-posterior length of eye, except for one vibrissa achieving a maximum length similar to anterior-posterior length of eye.
			\paragraph{Scutellum}
				Narrowly exposed, with visible area approximately equal to length of appressed scales, margins straight.
			\paragraph{Pleurites}
				Metepisternum nearly hidden by elytron except for triangular extension.
			\paragraph{Thoracic sterna} 
				Mesocoxal cavities separated by 1/3\x width of mesocoxal cavity. 
				Metasternum with transverse sulcus not apparent; metacoxal cavities widely separated by ca. 2\x their width.
			\paragraph{Legs}
				Profemur/pronotum length ratio \#; profemur with distal 1/5 produced ventrally as a sub-rectangular projection covering tibial joint; condyle of tibial articulation occupying 4/5 of distal surface and 1/5 length of femur. 
				Protibia/profemur length ratio \#; protibial apex with ventral setal comb recessed in a subtly incurved groove; mucro present as a large, black, sub-triangular, medially-projected tooth, which is approximately equilateral and whose sides are subequal in length to surrounding setae. 
				Protarsus with tarsomere III 2\x as long as II; wider than long. 
				Metatibial apex with almond shaped convexity ringed by 8-9 short, spiniform setae.
			\paragraph{Elytra}
				Length/width ratio \#; widest at anterior 1/3-1/4; anterior margins jointly 1.5\x wider than posterior margin of pronotum; lateral margins subparallel to slightly rounded after anterior 1/3, more strongly rounded and converging in posterior 1/3. 
				Posterior declivity angled at 70-85\td to main body axis. Elytra with 10 complete striae; striae shallow; punctures faint beneath appressed scales, separated by 5-7\x their diameter; intervals very slightly elevated.
			\paragraph{Abdominal sterna}
				Ventrite III anteromesally incurved around a fovea located mesally on anterior margin, posterior margin elevated and set off from IV along lateral 1/3s of its length. 
				Sternum VII mesally 1/2\x as long as wide; setae darkening, lengthening, and becoming more erect in posterior 2/3; anterior margin weakly curved.
			\paragraph{Tergum}
				Pygidium (tergum VIII) sub-cylindrical; medial 1/3 of anterior 2/3 of pygidum less sclerotized.
			\paragraph{Sternum VIII}
				Anterior laminar edges each incurved forming a 140\td angle with lateral margin; slightly less sclerotized medially between arms of bifurcation; posterior edge subtly incurved medially.
			\paragraph{Ovipositor}
				Coxites 1.5\x as long as broad, glabrous; styli 1/2\x as long as coxites. Genital chamber apically sclerotized.
			\paragraph{Spermatheca}
				Comma-shaped; collum short, apically with a large, hood-shaped projection angled at ca. 60\td to ramus, nearly equal in length and contiously aligned with curvature of bulb of ramus; collum sub-contiguous with, and angled at 90\td to ramus; ramus elongate, sub-cylindrical to slightly bulbous, 4/5\x thickness of corpus; corpus swollen, 1.25\x thicknes of ramus and 1.5\x thickness of cornu; cornu elongate, strongly recurved in basal 1/3, nearly straight thereafter and narrowing apically, abruptly narrowed in apical 1/3 with apex angled at 30\td to corpus.
		\subsubsection*{Description of male}
			Similar to female, except where noted.
			\paragraph{Habitus}
				Length \# mm, width \# mm, length/width ratio \#. Rostrum length \# mm, rostrum/pronotum length ratio \#, rostrum length/width ratio \#. Pronotum length/width ratio \#. Profemur/pronotum length ratio \#, protibia/profemur length ratio \#. Elytra length/width ratio \#.
			\paragraph{Elytra}
				Elytral declivity more angulate than female on average, forming an 80\td angle to main body axis, but otherwise as in female.
			\paragraph{Abdominal sterna}
				Sternum VII 2/5-1/2 \x as long as wide, posterior margin arcuate mesally.
			\paragraph{Tergum}
				Pygidium (tergum VIII) with posterior 1/3 punctate; anterior 2/3 rugose.
			\paragraph{Sternum IX}
				Spiculum gastrale 2\x length of aedeagal pedon. Laminar alae located on lateral 1/4 of posterior margin.
			\paragraph{Aedeagus}
				Length/width ratio \#; lateral margins very slightly converging posteriorly, abruptly constricted and more strongly converging in apical 1/5.
				Pedon in lateral view becoming gradually narrower posteriorly in anterior 1/2, ventral margins in posterior 1/2 abruptly curving to meet dorsal margins at a rounded apical point.
				Flagellum with large, elonage, tortuous apical sclerite, sclerite nearly as long as pedon, with complex, asymmetrical interior structure.
		\subsubsection*{Etymology}
			Named in reference to the long, somewhat unkempt, erect setae on the anterior margin of the pronotum; franko = free; Old High-German adjective \citep{brown1956}.
		\subsubsection*{Material examined}
			\paragraph{Holotype}
				\female~``MEX: S.L.P 1 km N.; Entronque El Huizache; 1493 m 2.VI.87; R. Anderson, \underline{\smash{Sphaeralcea}}; \underline{\smash{hastula}} A. Gray'' (\texttt{\textbf{CMNC}}).
			\paragraph{Paratypes}
				Same label information as female holotype (\texttt{\textbf{CMNC}}:  1 \female, 1 \male; \texttt{\textbf{TAMU}}: 2 \male);
				``MEXICO: S.L.P; 19.6 mi. n. Huizache; July 25, 1976; Peigler, Gruetzmacher,; R\&M Murray, Schaffner'' (\texttt{\textbf{CMNC}}: 1 \male);
				``MEXICO: San Luis Potosi; Entronque el Hulzache; 2 June 1987; R. Turnbow'' (\texttt{\textbf{CMNC}}: 1 \female, 1 \male);
				``MEXICO:Tamaulipas; 8.8 mi. ne. Jaumave; October 10, 1973; Gaumer \& Clark'' (\texttt{\textbf{TAMU}}: 2\female);
				``9 mi east Santo; Domingo, S.L.P.,; Mexico~~~XI-14-68; Veryl V. Board'' (\texttt{\textbf{TAMU}}: 2 \male).
		\subsubsection*{Distribution}
			\#
		\subsubsection*{Natural history}
			Associated with spear globemallow [\textit{Sphaeralcea hastulata} A. Gray; Malvaceae].

	\subsection*{\textit{Minyomerus sculptilis} - sp. n.}
		\subsubsection*{Diagnosis}
		\subsubsection*{Description of female}
			\paragraph{Habitus}
				Length \# mm, width \# mm, length/width ratio \#, widest at anterior 1/5 of elytra.
				Integument orange-brown to black.
				Scales with variously interspersed colors ranging from slightly off-white or beige to golden brown to dark coffee brown.
				Setae sub-recumbent to sub-erect, white to brown in color.
			\paragraph{Manidbles}
				Covered with white scales, with 3 longer setae, and 1 shorter seta between these.
			\paragraph{Rostrum}
				Length \# mm, anterior portion \# broader than long, rostrum/pronotum length ratio \#, rostrum length/width ratio \#.
				Separation of rostrum from head generally obscure. 
				Dorsal outline of rostrum nearly square, anterior half of dorsal surface mesally concave, posterior half coarsely but shallowly punctate to rugose. 
				Rostrum in lateral view nearly square; apical margin bisinuate and emarginate, with 2 pairs of large vibrissae. 
				Nasal plate defined by Y-shaped, impressed lines, convex, integument covered with white scales.
				Margins of mandibular incision directed \#\td outward dorsally in frontal view. 
				Ventrolateral sulci strongly defined, beginning as a narrow sulcus posteriad of insertion point of mandibles, running parallel to scrobe, terminating in a ventral fovea.
			\paragraph{Antennae}
				Dorsal margin of scrobe overhanging broadly (not forming a minute tooth).
				Funicle slightly longer than scape.
				Scape extending Brassicaceae  to posterior 1/4 of eye.
				Club nearly 3\x as long as wide.
			\paragraph{Head}
				Eyes globular, anterodorsal margin of each eye impressed, posterior margin slightly elevated from lateral surface of head; eyes separated in dorsal view by 5\x their anterior-posterior length, set off from anterior prothoracic margin by 1/4 of their anterior-posterior length.
				Head between eyes rugose and slightly bulging.
			\paragraph{Pronotum}
				Length/width ratio \#; widest near anterior 2/5.
				Anterior margin arcuate, subtly incurved mesally, and somewhat produced dorsally; anterior constriction broad, posterior margin slightly arcuate.
				Pronotum in lateral view with setae that reach beyond anterior margin; these setae becoming slightly longer and more erect laterally.
				Anterolateral margin with a reduced tuft of 3-6 post-ocular vibrissae present, emerging near ventral 1/2 of eye, and stopping just below ventral margin of eye; vibrissae varying in length from 1/2\x anterior-posterior length of eye to a maximum length similar to anterior-posterior length of eye.
			\paragraph{Scutellum}
				Exposed, margins straight.
			\paragraph{Pleurites}
				Metepisternum nearly hidden by elytron except for triangular extension.
			\paragraph{Thoracic sterna} 
				Mesocoxal cavities separated by 1/3\x width of mesocoxal cavity. 
				Metasternum with transverse sulcus not apparent; metacoxal cavities widely separated by ca. 2\x their width.
			\paragraph{Legs}
				Profemur/pronotum length ratio \#; profemur with distal 1/6 produced ventrally as a slightly rounded, sub-rectangular projection covering tibial joint; condyle of tibial articulation occupying 4/5 of distal surface and 1/6 length of femur. 
				Protibia/profemur length ratio \#; protibial apex with ventral setal comb recessed in a subtly incurved groove; mucro not apparent. 
				Protarsus with tarsomere III 1.5\x as long as II; wider than long. 
				Metatibial apex with almond shaped convexity ringed by 10-12 short, spiniform setae.
			\paragraph{Elytra}
				Length/width ratio \#; widest at anterior 1/5; anterior margins jointly 1.5-2\x wider than posterior margin of pronotum; lateral margins gently converging after anterior 1/5, more strongly converging in posterior 1/4. 
				Posterior declivity angled at 65-70\td to main body axis. Elytra with 10 complete striae; striae broadly sculpted; punctures faint beneath appressed scales, separated by 5-7\x their diameter; intervals elevated, with every second interval, beginning at elytral suture, more strongly raised than adjacent intervals.
			\paragraph{Abdominal sterna}
				Ventrite III anteromesally incurved around a fovea located mesally on anterior margin, posterior margin elevated and set off from IV along lateral 1/3s of its length. 
				Sternum VII mesally 2/3\x as long as wide; anterior margin straight.
			\paragraph{Tergum}
				Pygidium sub-cylindrical; medial 1/2 of anterior 3/5 of pygidium less sclerotized.
			\paragraph{Sternum VIII}
				Anterior laminar edges of spiculum ventrale each incurved forming a 125\td angle with lateral margin; lamina more sclerotized medially; posterior margin medially incurved.
			\paragraph{Ovipositor}
				Coxites as long as broad; styli as long as coxites, glabrous.
			\paragraph{Spermatheca}
				S-shaped; collum short, apically with a large, hood-shaped projection roughly aligned with central axis of corpus, nearly equal in length to bulb of ramus; collum sub-contiguous with, and angled at 30\td to ramus; ramus elongate, sub-cylindrical to slightly bulbous, 3/4\x thickness of corpus, with a short stalk oriented at ca. 45\td to the corpus; corpus swollen, 1.3\x thicknes of ramus; cornu short, 2.5-3\x length or ramus, recurved and strongly arched in basal 1/2, forming an inner angle of ca. 80\td, feebly sinuate thereafter, with apical 1/2 expanded, then abruptly constricted near apical 1/4 to a fine point.
		\subsubsection*{Description of male}
			Similar to female, except where noted.
			\paragraph{Habitus}
				Length \# mm, width \# mm, length/width ratio \#. Rostrum length \# mm, rostrum/pronotum length ratio \#, rostrum length/width ratio \#. Pronotum length/width ratio \#. Profemur/pronotum length ratio \#, protibia/profemur length ratio \#. Elytra length/width ratio \#.
			\paragraph{Elytra}
				Elytral declivity slightly less angulate than female, forming a 60\td angle to main body axis, but otherwise as in female.
			\paragraph{Abdominal sterna}
				Sternum VII 1/2 \x as long as wide, posterior margin feebly arcuate mesally.
			\paragraph{Tergum}
				Pygidium (tergum VIII) with mesal 1/3 of posterior margin subtly incurved; posterior 2/3 punctate; anterior 1/3 rugose.
			\paragraph{Sternum VIII}
				Consisting of 2 sub-triangular sclerites; antero-laterally with a sharply-pointed projection as long as anterior-posterior length of triangular portion of sclerite.
			\paragraph{Aedeagus}
				Length/width ratio \#; lateral margins parallel, more strongly converging in region of ostium. 
				In lateral view, width of pedon even throughout in anterior 2/3, ventral margins in posterior 1/3 becoming straight towards apex, then curving to meet dorsal margins at a sharp apical point; apex acutely angulate. 
				Flagellum without apparent sclerite.
		\subsubsection*{Comments}
			Due to the limited number of specimens of this species, dissections of mouthparts could not be performed.
		\subsubsection*{Etymology}
			Named in reference to the elevated elytral intervals, which give this species a sculpted appearance; sculptilis = sculpted; Latin adjective \citep{brown1956}.
		\subsubsection*{Material examined}
			\paragraph{Holotype}
				\female~``Burley, Idaho; \#7, 5-20-32; \underline{A.} \underline{tridentata}; David E. Fox'' (\texttt{\textbf{USNM}}).
			\paragraph{Paratypes}
				``Milner, Idaho; \#5a, 7-9-31; S. pestifer; David E. Fox'' (\texttt{\textbf{USNM}}: 1 \female);
				``Hazelton, Ida; \#10~~~4/29/30; N. altissma'' (\texttt{\textbf{USNM}}: 1 \male)
		\subsubsection*{Distribution}
			\#
		\subsubsection*{Natural history}
			Associated with big sagebrush [\textit{Artemisia tridentata} Nutt.; Asteraceae], tumbleweed [\textit{Salsola tragus} L.; Amaranthaceae], tall tumblemustard [\textit{Sisymbrium altissimum} L.; Brassicaceae].

	\subsection*{\textit{Minyomerus tylotos} - sp. n.}
		\subsubsection*{Diagnosis}
		\subsubsection*{Description of female}
			\paragraph{Habitus}
				Length \# mm, width \# mm, length/width ratio \#, widest at anterior 1/6 of elytra.
				Integument orange-brown to black. 
				Scales with variously interspersed colors ranging from slightly off-white or beige to manila/tan to dark coffee brown, in some specimens appearing semi-translucent (in others opaque).
				Setae linear to apically explanate, appearing minutely spatulate, sub-recumbent to sub-erect, tan to brown in color.
			\paragraph{Manidbles}
				Covered with white scales, with 2-3 longer setae, and 1-3 shorter setae between these.
			\paragraph{Maxillae}
				Cardo bifurcate at base with an inner angle of ca. 90\td, arms roughly equal in length and width, arms of bifurcation equal in length to apically outcurved arm. 
				Stipes subrectangular, 1.5\x wider than long, roughly equal in width to inner arm of bifurcation of cardo, glabrous. 
				Galeo-lacinial complex nearly extending to apex of maxillary palpomere I; complex mesally membranous, laterally sclerotized, with sharp demarcation of sclerotized region separating palpiger from galeo-lacinial complex; setose in membranous area just adjacent to sclerotized region, setae covering 1/2 of dorsal surface area; dorsally with 5 apicomesal lacinial teeth; ventrally with 3 reduced lacinial teeth. 
				Palpiger with a single lateral seta, otherwise glabrous, anterior 1/2 membranous, posteriorly sclerotized.
			\paragraph{Maxillary palps}
				I apically oblique, apical end forming a 45\td angle with base, with 2 apical setae; II sub-cylindrical, with 1 apical seta.
			\paragraph{Labium}
				Prementum roughly pentagonal; apical margins arcuate, medially angulate; lateral margins feebly incurved; basal margin arcuate. 
				Labial palps 3-segmented, I with apical 1/2 projecting beyond margin of prementum, reaching apex of ligula; III slightly
longer than II.
			\paragraph{Rostrum}
				Length \# mm, anterior portion \# broader than long, rostrum/pronotum length ratio \#, rostrum length/width ratio \#.
				Separation of rostrum from head generally obscure. 
				Dorsal outline of rostrum nearly square, anterior half of dorsal surface feebly impressed, posterior half coarsely but shallowly punctate to rugose. 
				Rostrum in lateral view nearly square; apical margin strongly bisinuate and emarginate, appearing medially notched, with 2 large vibrissae. 
				Nasal plate lacking distinct impressions, having instead a poorly defined anteromesal convexity, integument completely and evenly covered with white scales. 
				Margins of mandibular incision directed \#\td outward dorsally in frontal view. 
				Ventrolateral sulci weakly defined as a broad concavity dorsad of insertion point of mandibles, running parallel to scrobe, becoming flatter posteriorly and disappearing ventrally.
				Dorsal surface of rostrum with median fovea short and linear, or punctate.
				Rostrum ventrally with subparallel sulci beginning at corners of oral cavity and continuing halfway to back of head.
			\paragraph{Antennae}
				Minute tooth formed by overhanging dorsal margin of scrobe anterior to margin of eye by 1/3 of length of eye.
				Scape extending to posterior margin of eye.
				Terminal funicular antennomere lacking appressed scales, having instead a covering of apically-directed pubescence with interspersed sub-erect setae.
				Club nearly 3\x as long as wide.
			\paragraph{Head}
				Eyes globular and somewhat elongate, strongly impressed, slanted ca. 45\td antero-ventrally; eyes separated in dorsal view by 4\x their anterior-posterior length, set off from anterior prothoracic margin by 1/4 of their anterior-posterior length. 
				Head between eyes punctate and protuberant.
			\paragraph{Pronotum}
				Length/width ratio \#; widest near anterior 2/5; somewhat globular. 
				Anterior margin arcuate, but feebly incurved mesally, lateral margins evenly curved and widening into a bulge just anteriad of midpoint of pronotum, posterior margin straight, with a slight mesal incurvature. 
				Pronotum in lateral view with transverse ventrolateral sulci strongly excavated and distinctly sculptured; with short, recumbent to suberect setae that barely attain or reach just beyond anterior margin; these setae becoming shorter and more erect laterally, reaching a maximum length nearly equal to length of eye; dorsally, these setae become uniquely apically explanate, with a longitudinal, medial, ridgelike portion that tapers to either side apicolaterally.
				Anterolateral margin with a single ocular vibrissa present, emerging near ventral margin of eye; vibrissa achieving a maximum length of 2/5 of anterior-posterior length of eye.
			\paragraph{Scutellum}
				Not exposed.
			\paragraph{Pleurites}
				Metepisternum nearly hidden by elytron except for triangular extension.
			\paragraph{Thoracic sterna} 
				Mesocoxal cavities separated by 1/3\x width of mesocoxal cavity. 
				Metasternum with transverse sulcus not apparent; metacoxal cavities widely separated by ca. 3\x their width.
			\paragraph{Legs}
				Profemur/pronotum length ratio \#; profemur with distal 1/5 produced ventrally as a sub-rectangular projection covering tibial joint; condyle of tibial articulation occupying 4/5 of distal surface and 1/5 length of femur. 
				Protibia/profemur length ratio \#; protibial apex with ventral setal comb recessed in a subtly incurved groove; mucro present as an acute, medially-projected tooth, which is approximately equal in length to surrounding setae. 
				Protarsus with tarsomere III 2\x as long as II; wider than long. 
				Metatibial apex with weakly projecting, poorly defined, narrow convexity laterally flanged by 5 short, spiniform setae.
			\paragraph{Elytra}
				Length/width ratio \#; widest at anterior 1/6; anterior margins jointly 1.5-2\x wider than posterior margin of pronotum; lateral margins nearly straight and subparallel after anterior 1/5, converging in posterior 1/3. 
				Posterior declivity angled at 70-75\td to main body axis. Elytra with 10 complete striae; striae broadly sculpted; punctures broad and faint beneath appressed scales, separated by 4-5\x their diameter; intervals elevated.
			\paragraph{Abdominal sterna}
				Ventrite III anteromesally incurved around a fovea located mesally on anterior margin, posterior margin elevated and set off from IV along lateral 3/8s of its length. 
				Sternum VII mesally 2/3\x as long as wide; setae slightly lengthening, and becoming medially directed in posterior 1/3; anterior margin weakly curved; posterior margin distinctly incurved mesally, appearing broadly notched; surface of sternite concave, appearing broadly foveate, immediately anteriad of marginal incurvature.
			\paragraph{Tergum}
				Tergum VII mesally incurved.
				Pygidium sub-cylindrical; medial 1/3 of anterior 2/3 of pygidium less sclerotized, with a patch of very short, fine setae.
			\paragraph{Sternum VIII}
				Anterior laminar edges each incurved forming a 130\td angle with lateral margin; slightly less sclerotized medially between arms; posterior margin medially incurved.
			\paragraph{Ovipositor}
				Coxites as long as broad; styli with 3 setae near the base.
			\paragraph{Spermatheca}
				?-shaped; collum short, apically with a large, angulate, hood-shaped projection angled at 45\td to corpus, sub-equal in length to ramus and contiously aligned with curvature of bulb of ramus; collum sub-contiguous with, and angled at ca. 60\td to ramus; ramus basally elongate and constricted, forming a stalk, 1/3\x length of collum, bulbous apically, 3\x thicker than stalk; corpus not swollen, of equal thickness to collum and cornu; cornu elongate, apically, gradually narrowed, strongly recurved in basal 1/3, straight along mesal 1/3, and curved near apical 1/3 such that apex is parallel to collum and corpus.
		\subsubsection*{Description of male}
			Not available or known.
		\subsubsection*{Comments}
		\subsubsection*{Etymology}
			Named in reference to the short, apically explanate setae interspersed throughout the dorsum, which give this species a distinctly ``knobbed'' appearance; tylotos = knobby; Greek adjective \citep{brown1956}.
		\subsubsection*{Material examined}
			\paragraph{Holotype}
				\female~``H. O. Canyon,; Davis Mts., Texas; Jeff Davis County; VII-20-1968, 6200'; J. E. Hafernik'' (\texttt{\textbf{TAMU}}).
			\paragraph{Paratypes}
				``24 mi. wsw. Ft. Davis; Jeff Davis Co., Texas; August 17, 1969; Board \& Hafernik'' (\texttt{\textbf{TAMU}}: 1 \female);
				``USA Texas Jeff Davis Co.; 4.1 mi. S. Fort Davis; sweeping grasses-weeds; 4750' . 19.VII.82; R.S. Anderson'' (\texttt{\textbf{CMNC}}: 1 \female)
		\subsubsection*{Distribution}
			\#
		\subsubsection*{Natural history}
			No host plant associations have been documented.
			It is unknown whether this species is parthenogenetic.
			
\section*{Checklist of species}
	\#
\section*{Identification key}
	\#
\section*{Phylogenetic placement of new species}
	\#
\section*{Acknowledgments}

\newpage
\bibliography{references} 
%just references in windows, use references.bib in linux
\end{document}
